\documentclass[11pt]{article}
\usepackage[danish]{babel}


\begin{document}
\begin{titlepage}
\title{Krav til Bachelor}
\author{Mads Wulff Nielsen }
\maketitle
\thispagestyle{empty} % sørger for at sidetal er fjernet fra forside
\end{titlepage}
\tableofcontents
\thispagestyle{empty}
\clearpage
\section{Krav}
\subsection{Overordnede krav} 
I bachelorprojektet skal den studerende dokumentere evnen til på et
analytisk og metodisk grundlag at kunne bearbejde en kompleks og praksisnær
problemstilling i relation til en konkret opgave inden for IT-området.
\subsubsection{Rapportens omfang} 
40 + 20 normalsider pr studerende
\subsubsection{Sprog:} Dansk eller engelsk
\subsubsection{Forslag til sessioner i rapport}
\begin{itemize}
\item Forside, Title
\item Abstrakt
\item Introduktion
\item Teknologi
\item Krav specifikation og design
\item Udvikling og implementering
\item Konklusion
\item Bilag
\item Referencer
\end{itemize}
\clearpage

\section{Læringsmål}

Det afsluttende bachelorprojekt skal dokumentere, at uddannelsens afgangsniveau er opnået, jf. bilag 1 i BEK for professionsbacheloruddannelsen i softwareudvikling: \\
\subsubsection{Viden}
Den uddannede har viden om:
\begin{itemize}
\item den strategiske rolle af test i systemudvikling
\item globalisering af softwareproduktion
\item systemarkitektur og forståelse af dens strategiske betydning for virksomhedens forretning
\item anvendt teori og metode samt udbredte teknologier inden for domænet
\item sammenhænge mellem anvendt teori, metode og teknologi og kan reflektere over disses egnethed i forskellige situationer
\end{itemize}
\subsubsection{Færdigheder} 
Den uddannede kan:
\begin{itemize}
\item håndtere planlægning og gennemførelse af test af større IT-systemer
\item indgå professionelt i samarbejde omkring udvikling af store systemer ved anvendelse af udbredte metoder og teknologier
\item sætte sig ind i nye teknologier og standarder til håndtering af integration mellem systemer

\item gennem praksis udvikle egen kompetenceprofil fra primært at være en backend-udviklerprofil til at varetage opgaver som
systemarkitekt
\item håndtere fastlæggelse og realisering af en såvel forretningsmæssig som teknologisk hensigtsmæssig arkitektur for store systemer

\end{itemize}

\subsubsection{Kompetencer}

Den uddannede kan:
\begin{itemize}
\item håndtere planlægning og gennemførelse af test af større IT-systemer
\item indgå professionelt i samarbejde omkring udvikling af store systemer ved
anvendelse af udbredte metoder og teknologier
\item sætte sig ind i nye teknologier og standarder til håndtering af integration
mellem systemer
\item gennem praksis udvikle egen kompetenceprofil fra primært at være en
backend-udviklerprofil til at varetage opgaver som systemarkitekt
\item håndtere fastlæggelse og realisering af en såvel forretningsmæssig som
teknologisk hensigtsmæssig arkitektur for store systemer
\end{itemize}

\clearpage
\section{Litteratur}
\begin{thebibliography}{9}
\bibitem{link} 
    Copenhagen Business Academy, PBA SOFT Studieordning
    \\\texttt{https://www.cphbusiness.dk}
\end{thebibliography}

\end{document}